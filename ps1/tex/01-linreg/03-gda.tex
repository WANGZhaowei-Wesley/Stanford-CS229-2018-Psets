\clearpage
\item \subquestionpoints{5}
Recall that in GDA we model the joint distribution of $(x, y)$ by the following
equations:
%
\begin{eqnarray*}
	p(y) &=& \begin{cases}
	\phi & \mbox{if~} y = 1 \\
	1 - \phi & \mbox{if~} y = 0 \end{cases} \\
	p(x | y=0) &=& \frac{1}{(2\pi)^{n/2} |\Sigma|^{1/2}}
		\exp\left(-\frac{1}{2}(x-\mu_{0})^T \Sigma^{-1} (x-\mu_{0})\right) \\
	p(x | y=1) &=& \frac{1}{(2\pi)^{n/2} |\Sigma|^{1/2}}
		\exp\left(-\frac{1}{2}(x-\mu_1)^T \Sigma^{-1} (x-\mu_1) \right),
\end{eqnarray*}
%
where $\phi$, $\mu_0$, $\mu_1$, and $\Sigma$ are the parameters of our model.

Suppose we have already fit $\phi$, $\mu_0$, $\mu_1$, and $\Sigma$, and now
want to predict $y$ given a new point $x$. To show that GDA results in a
classifier that has a linear decision boundary, show the posterior distribution
can be written as
%
\begin{equation*}
	p(y = 1\mid x; \phi, \mu_0, \mu_1, \Sigma)
	= \frac{1}{1 + \exp(-(\theta^T x + \theta_0))},
\end{equation*}
%
where $\theta\in\Re^n$ and $\theta_{0}\in\Re$ are appropriate functions of
$\phi$, $\Sigma$, $\mu_0$, and $\mu_1$.

\ifnum\solutions=1{
  \begin{answer}
\begin{align*}
    P(y=1|x;\phi, \mu_0, \mu_1, \Sigma) &= \frac{P(y=1)P(x|y=1)}{P(y=0)P(x|y=0)+P(y=1)P(x|y=1)}\\
    &= \frac{\phi exp(-\frac{1}{2}(x-\mu_1)^T\Sigma^{-1}(x-\mu_1))}{\phi exp(-\frac{1}{2}(x-\mu_1)^T\Sigma^{-1}(x-\mu_1)) + (1 - \phi) exp(-\frac{1}{2}(x-\mu_0)^T\Sigma^{-1}(x-\mu_0))}\\
    &= \frac{1}{1 + \frac{1-\phi}{\phi}exp(-\frac{1}{2}(x-\mu_0)^T\Sigma^{-1}(x-\mu_0)+\frac{1}{2}(x-\mu_1)^T\Sigma^{-1}(x-\mu_1))}\\
    &=\frac{1}{1+exp(log\frac{1-\phi}{\phi} + \mu_0^T\Sigma^{-1}x - \mu_1^T\Sigma^{-1}x + \frac{1}{2}(\mu_1^T\Sigma^{-1}\mu_1 - \mu_0^T\Sigma^{-1}\mu_0)}
\end{align*}
So, the following formula are $\theta$ and $\theta_0$:
\begin{align*}
    \theta &= \Sigma^{-1}(\mu_1-\mu_0)\\
    \theta_0 &= log\frac{\phi}{1 - \phi} + \frac{1}{2}(\mu_0^T\Sigma^{-1}\mu_0 - \mu_1^T\Sigma^{-1}\mu_1)
\end{align*}
\end{answer}

}\fi
