\clearpage
\item \points{10} {\bf Model Calibration}

In this question we will try to understand the output $h_\theta(x)$ of the
hypothesis function of a logistic regression model, in particular why we might
treat the output as a probability (besides the fact that the sigmoid function
ensures $h_\theta(x)$ always lies in the interval $(0, 1)$).

When the probabilities outputted by a model match empirical observation, the
model is said to be \emph{well calibrated} (or reliable). For example, if we
consider a set of examples $x^{(i)}$ for which $h_\theta(x^{(i)})  \approx
0.7$, around 70\% of those examples should have positive labels. In a
well calibrated model, this property will hold true at every probability value.

Logistic regression tends to output well calibrated probabilities (this is
often not true with other classifiers such as Naive Bayes, or SVMs). We will
dig a little deeper in order to understand why this is the case, and find that
the structure of the loss function explains this property.

Suppose we have a training set $\{x^{(i)},y^{(i)}\}_{i=1}^m$ with
$x^{(i)} \in \mathbb{R}^{n+1}$ and $y^{(i)} \in \{0, 1\}$. Assume we have an
intercept term $x_0^{(i)} = 1$ for all $i$. Let $\theta \in \mathbb{R}^{n+1}$
be the maximum likelihood parameters learned after training a logistic
regression model. In order for the model to be considered well calibrated,
given any range of probabilities $(a, b)$ such that $0 \le a < b \le 1$, and
training examples $x^{(i)}$ where the model outputs $h_\theta(x^{(i)})$ fall in
the range $(a, b)$, the fraction of positives in that set of examples should be
equal to the average of the model outputs for those examples. That is, the
following property must hold:
$$
  \frac{\sum_{i\in I_{a,b}}  P\left(y^{(i)}=1|x^{(i)};\theta\right)}
       {{|\{i\in I_{a,b}\}|}}
  = \frac{\sum_{i\in I_{a,b}} \mathbb{I}\{y^{(i)} = 1\}}
         {|\{i\in I_{a,b}\}|},
$$
where $P(y=1|x;\theta) = h_\theta(x) = 1/(1+\exp(-\theta^\top x))$, $I_{a,b} =
\{ i | i \in \{1,...,m\},  h_\theta(x^{(i)}) \in (a,b)\} $ is an index set of
all training examples $x^{(i)}$ where $h_\theta(x^{(i)}) \in (a,b)$, and $|S|$
denotes the size of the set $S$.

\begin{enumerate}
  \item \subquestionpoints{5}
Show that the above property holds true for the described logistic regression
model over the range $(a,b) = (0,1)$.

\textit{Hint}: Use the fact that we include a bias term.

\ifnum\solutions=1 {
  \begin{answer}
note that we have following equations:
\begin{align*}
J(\theta) &= - \frac{1}{m} \sum \limits_{i=1}^m L_i(\theta)\\
L_i(\theta) &= y^{(i)}\log h_\theta(x^{(i)})
\end{align*}
We can take the derivative of $L_i(\theta)$ with respect to the k(th) element of $\theta$. The result is following:
\begin{align*}
\frac{\partial L_i(\theta)}{\partial \theta_k} &= y^{(i)}\frac{1}{h_\theta(x^{(i)})}h_\theta(x^{(i)})(1-h_\theta(x^{(i)}))x^{(i)}_k\\&\quad-(1-y^{(i)})\frac{1}{1-h_\theta(x^{(i)})}(1-h_\theta(x^{(i)}))h_\theta(x^{(i)})x^{(i)}_k\\
&=y^{(i)}x^{(i)}_k - y^{(i)}h_\theta(x^{(i)})x^{(i)}_k+y^{(i)}h_\theta(x^{(i)})x^{(i)}_k-h_\theta(x^{(i)})x^{(i)}_k\\
&=(y^{(i)}-h_\theta(x^{(i)}))x^{(i)}_k
\end{align*}
Then, we need to calculate the second derivative of $L_i(\theta)$
\begin{align*}
    \frac{\partial L_i(\theta)}{\partial \theta_k \partial \theta_j} &= \frac{\partial (y^{(i)}-h_\theta(x^{(i)}))x^{(i)}_k}{\partial \theta_j}\\
    &=x^{(i)}_k\cdot(-1)\cdot \frac{\partial g(\theta^T x^{(i)}}{\partial \theta_j}\\
    &=-x^{(i)}_k x^{(i)}_j h_\theta(x^{(i)}) (1 - h_\theta(x^{(i)}))
\end{align*}
Using the above equations, we can calculate the Hessian H of the fucntion $J(\theta)$. The element of H in row k and column j is following:
\begin{align*}
    \frac{\partial J(\theta)}{\partial \theta_k \partial \theta_j} &= - \frac{1}{m}\sum\limits_{i=1}^m -x^{(i)}_k x^{(i)}_j h_\theta(x^{(i)}) (1 - h_\theta(x^{(i)}))\\
    &= \frac{1}{m}\sum\limits_{i=1}^m x^{(i)}_k x^{(i)}_j h_\theta(x^{(i)}) (1 - h_\theta(x^{(i)}))
\end{align*}
So, take a random vector $z \in R^n$, $z^T H z$ is following:
\begin{align*}
    z^T H z &= \frac{1}{m}\sum\limits_{i=1}^m h_\theta(x^{(i)}) (1 - h_\theta(x^{(i)})) \sum\limits_j \sum\limits_k z_j x^{(i)}_j x^{(i)}_k z_k\\
    &= \frac{1}{m}\sum\limits_{i=1}^m h_\theta(x^{(i)}) (1 - h_\theta(x^{(i)})) \sum\limits_j (z_j x^{(i)}_j)^2 >= 0
\end{align*}
\end{answer}

} \fi

  \item \subquestionpoints{3}
If we have a binary classification model that is perfectly calibrated---that
is, the property we just proved holds for any $(a, b) \subset [0, 1]$---does
this necessarily imply that the model achieves perfect accuracy? Is the
converse necessarily true? Justify your answers.

\ifnum\solutions=1 {
  \begin{answer}
At first, I think the description about well calibrated model is not accurate. If we choose an interval $(a, b)$ whose $\{i \in I_{a, b}\}$ only has positive examples(or negative samples). It's impossible for $\frac{\sum_{i\in I_{a, b}}P(y^{(i)}=1|x^{(i)};\theta)}{|\{i\in I_{a,b}\}|} = \frac{\sum_{i\in I_{a, b}}\mathbb{I}\{y^{(i)}=1\}}{|\{i\in I_{a, b}\}|}$ to hold true, because $P(y^{(i)}=1|x^{(i)};\theta)$ can't be 1(or 0). Thus, no model is perfectly calibrated.\\ \\
For a binary classification model, we should only consider intervals whose $\{i \in I_{a, b}\}$ has both positive and negative examples. If we consider a set $S$ of only one positive example and only one negative example, it's clear that a model that is perfectly calibrated doesn't imply that the model achieves perfect accuracy. For example, the model that outputs 0.8 for the negative example and 0.2 for the positive example is well calibrated, but it doesn't achieve perfect accuracy.\\ \\
Conversely, perfect accuracy doesn't lead to perfect calibration. We consider a train set of one positive example and one negative example again. If the model we got by training outputs 0.6 for the positive one and 0.1 for the negative one, the model achieves perfect accuracy. However, the property doesn't hold true.
\end{answer}

} \fi

  \item \subquestionpoints{2}
Discuss what effect including $L_2$ regularization in the logistic regression
objective has on model calibration.

\ifnum\solutions=1 {
  \begin{answer}
By add an regularization term, the loss function becomes $$J(\theta) = \lambda||\theta||^2 + \sum (y\log h_\theta(x)+(1-y)\log(1-h_\theta(x)))$$
The derivative of the loss function becomes $$J'(\theta) = 2\lambda\theta + \sum(y - h_\theta(x))x $$
set the derivative to zero
$$2\lambda \begin{bmatrix}\theta_0\\ \theta_1\\ \theta_2\\ ...\\ \theta_n\end{bmatrix} + \sum(y - h_\theta(x))\begin{bmatrix}1\\ x_1\\ x_2\\ ...\\ x_n\end{bmatrix} = 0$$
consider the first row
\begin{align*}
    &2\lambda\theta_0 + \sum(y - h_\theta(x)) = 0\\
    &\sum h_\theta(x) = 2\lambda\theta_0 + \sum \mathbb{I}\{y = 1\}
\end{align*}
So, the prediction is biased by a constant $2\lambda\theta_0$
\end{answer}

} \fi

\end{enumerate}

\textbf{Remark:}  We considered the range $(a,b) = (0, 1)$. This is the only
range for which logistic regression is guaranteed to be calibrated on the
training set. When the GLM modeling assumptions hold, all ranges $(a,b) \subset
[0,1]$ are well calibrated. In addition, when the training and test set are
from the same distribution and when the model has not overfit or underfit,
logistic regression tends to be well calibrated on unseen test data as well.
This makes logistic regression a very popular model in practice, especially
when we are interested in the level of uncertainty in the model output.
