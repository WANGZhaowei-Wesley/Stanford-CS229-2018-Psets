\item \subquestionpoints{9} Let $K$ be a Mercer kernel corresponding to some
very high-dimensional feature
mapping $\phi$. Suppose $\phi$ is so high-dimensional (say,
$\infty$-dimensional) that it's infeasible to ever represent $\phi(x)$
explicitly.  Describe how you would apply the ``kernel trick'' to the
perceptron to make it work in the high-dimensional feature space $\phi$, but
without ever explicitly computing $\phi(x)$.

[\textbf{Note:} You don't have to worry about the intercept term.  If you like,
think of $\phi$ as having the property that $\phi_0(x) = 1$ so that this is
taken care of.] Your description should specify:
\begin{enumerate}[label=\roman*.]
  \item \subquestionpoints{3} How you will (implicitly) represent the
  high-dimensional
    parameter vector $\theta^{(i)}$, including how the initial value
    $\theta^{(0)} = 0$ is represented (note that $\theta^{(i)}$ is
    now a vector whose dimension is the same as the feature vectors
    $\phi(x)$);
  \item \subquestionpoints{3} How you will efficiently make a prediction on a
  new input
    $x^{(i+1)}$.  I.e., how you will compute
    $h_{\theta^{(i)}}(x^{(i+1)}) = g({\theta^{(i)}}^T \phi(x^{(i+1)}))$,
    using your representation of $\theta^{(i)}$; and
  \item \subquestionpoints{3} How you will modify the update rule given above
  to perform an
  update to $\theta$ on a new training example $(x^{(i+1)}, y^{(i+1)})$;
  \emph{i.e.,} using the update rule corresponding to the feature mapping
  $\phi$:
  \begin{equation*}
  \theta^{(i+1)} :=
	  \theta^{(i)} + \alpha (y^{(i+1)} - h_{\theta^{(i)}}(x^{(i+1)})) \phi(x^{(i+1)})
  \end{equation*}
\end{enumerate}

\ifnum\solutions=1 {
  \begin{answer}\\
a) The way to represent the high-dimensional parameter vector $\theta^{(i)}$: assume we have $m$ examples, we will represent $\theta^{(i)}$ as a linear combination of the feature vectors of $m$ examples. $$\theta^{(i)} = \sum\limits_{j=1}^m \beta_j \phi(x^{(j)})$$ \textbf{where $\beta_j = 0$ if $j > i$, otherwise $\beta_j = \alpha (y^{(j)} - h_{\theta^{(j-1)}} (x^{(j)}))$}\\
The $\theta^{(0)}$ is represented as a linear combination of the feature vectors of examples, where all $\beta_j = 0$ \\
b) The most difficult part of a prediction is how to compute $\theta^{{(i)}^T} \phi(x^{(i+1)})$. We can represent $\theta^{(i)}$ as a linear combination of the feature vectors of first $i$ training examples.
$$\theta^{{(i)}^T} \phi(x^{(i+1)}) = (\sum\limits_{k=1}^i \beta_k \phi(x^{(k)}))^T \phi(x^{(i+1)}) = \sum\limits_{k=1}^i \beta_k K(\phi(x^{(k)}), \phi(x^{(i+1)}))$$
So, we can use kernel $K$ to efficiently predict the label of a new input $x^{(i+1)}$.\\
c) we represent $\theta^{(i)}$ as a liner combination of $x^{(1)}-x^{(i)}$. So, we only need to use a vector $\lambda^{(i)}$ record $\alpha(y^{(j)} - h_{\theta^{(j-1)}}(x^{(j)}))$ for each $x^{(j)}$, where $j\leq i$.\\
The update rule is $\lambda^{(i)} = [\lambda^{(i-1)}, \alpha(y^{(i)} - h_{\theta^{(i-1)}}(x^{(i)}))]$. In othe words, we only need append a scalar $\alpha(y^{(i)} - h_{\theta^{(i-1)}}(x^{(i)}))$ at the end of $\lambda^{(i-1)}$ to get $\lambda^{(i)}$
\end{answer}

} \fi
