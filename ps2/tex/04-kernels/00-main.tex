\clearpage
\item \points{18} {\bf Constructing kernels}

In class, we saw that by choosing a kernel $K(x,z) = \phi(x)^T\phi(z)$, we can
implicitly map data to a high dimensional space, and have the SVM algorithm
work in that space.  One way to generate kernels is to explicitly define the
mapping $\phi$ to a higher dimensional space, and then work out the
corresponding $K$.

However in this question we are interested in direct construction of kernels. 
I.e., suppose we have a function $K(x,z)$ that we think gives an appropriate
similarity measure for our learning problem, and we are considering plugging
$K$ into the SVM as the kernel function. However for $K(x,z)$ to be a valid
kernel, it must correspond to an inner product in some higher dimensional space
resulting from some feature mapping $\phi$.  Mercer's theorem tells us that
$K(x,z)$ is a (Mercer) kernel if and only if for any finite set $\{x^{(1)},
\ldots, x^{(m)}\}$, the square matrix $K \in \Re^{m \times m}$ whose entries
are given by $K_{ij} = K(x^{(i)},x^{(j)})$ is symmetric and positive
semidefinite. You can find more details about Mercer's theorem in the notes,
though the description above is sufficient for this problem.

Now here comes the question: Let $K_1$, $K_2$ be kernels over $\Re^n \times
\Re^n$, let $a \in \Re^+$ be a positive real number, let $f : \Re^n \mapsto
\Re$ be a real-valued function, let $\phi: \Re^n \rightarrow \Re^d$ be a
function mapping from $\Re^n$ to $\Re^d$, let $K_3$ be a kernel over $\Re^d
\times \Re^d$, and let $p(x)$ a polynomial over $x$ with \emph{positive}
coefficients.

For each of the functions $K$ below, state whether it is necessarily a
kernel.  If you think it is, prove it; if you think it isn't, give a
counter-example.

\begin{enumerate}

\item \subquestionpoints{1} $K(x,z) = K_1(x,z) + K_2(x,z)$
\item \subquestionpoints{1} $K(x,z) = K_1(x,z) - K_2(x,z)$
\item \subquestionpoints{1} $K(x,z) = a K_1(x,z)$
\item \subquestionpoints{1} $K(x,z) = -a K_1(x,z)$
\item \subquestionpoints{5} $K(x,z) = K_1(x,z)K_2(x,z)$ 
\item \subquestionpoints{3} $K(x,z) = f(x)f(z)$
\item \subquestionpoints{3} $K(x,z) = K_3(\phi(x),\phi(z))$
\item \subquestionpoints{3} $K(x,z) = p(K_1(x,z))$

\end{enumerate}

[\textbf{Hint:} For part (e), the answer is that $K$ \emph{is} indeed
a kernel. You still have to prove it, though.  (This one may be harder than the
rest.)  This result may also be useful for another part of the problem.]

\ifnum\solutions=1 {
  \begin{answer}\\
Let's take a finite set $\{x^{(1)}, x^{(2)}, ..., x^{(m)}\}$, and we will build kernel matrix on this data set. 
\begin{enumerate}
\item $K(x,z) = K_1(x,z) + K_2(x,z)$\\ 
We already know $K_1$ and $K_2$ are kernels. So the kernel matrices $K_1$ and $K_2$ are symmetric and positive semidefinite.
For the i th and j th examples in the data set, it's easy to see $K_{ij} = K_{1 ij}+K_{2 ij}$. So, for kernel matrices, $K = K_1 + K_2$.\\
i) Because $K_1 = K_1^T$ and $K_2 = K_2^T$, we can know $K^T = (K_1 + K_2)^T = K_1^T + K_2^T = K_1 + K_2 = K$. So, the kernel matrix $K$ is symmetric.\\
ii) Because $\forall z \in \mathbb{R}^n, z^T K_1 z \geq 0\ and\ z^T K_2 z \geq 0$, we can get $\forall z \in \mathbb{R}^n, z^T K z = z^T (K_1 + K_2) z = z^T K_1 z + z^T K_2 z \geq 0$. So, the kernel matrix $K$ is positive semidefinite.\\
Based on i) and ii), we can know the kernel matrix $K$ is symmetric and positive semidefinite, the kernel function $K$, it is \textbf{necessarily} a kernel.
\item $K(x,z) = K_1(x,z) - K_2(x,z)$\\
It's easy to see, for kernel matrices, $K = K_1 - K_2$.\\
i) Because $\forall z \in \mathbb{R}^n, z^T K_1 z \geq 0\ and\ z^T K_2 z \geq 0$, we can get $\forall z \in \mathbb{R}^n, z^T K z = z^T (K_1 - K_2) z = z^T K_1 z - z^T K_2 z$.\\
So, the kernel function $K$ is \textbf{not necessarily} a kernel.\\
Counter Example:\\
$K_1 = \begin{bmatrix}1 & 0 \\ 0 & 1\end{bmatrix}, K_2 = \begin{bmatrix}2 & 0 \\ 0 & 2\end{bmatrix}, K = \begin{bmatrix}-1 & 0 \\ 0 & -1\end{bmatrix}$. K is not positive semidefinite.
\item $K(x,z) = a K_1(x,z)$\\
It's easy to see, for kernel matrices, $K = a K_1$.\\
i) Because $K_1 = K_1^T$, we can know $K^T = (a K_1)^T = a K_1^T = a K_1 = K$. So, the kernel matrix $K$ is symmetric.\\
ii) Because $\forall z \in \mathbb{R}^n, z^T K_1 z \geq 0$, we can get $\forall z \in \mathbb{R}^n, z^T K z = z^T (a K_1) z = a (z^T K_1 z)\geq 0$. So, the kernel matrix $K$ is positive semidefinite.\\
Based on i) and ii), we can know the kernel matrix $K$ is symmetric and positive semidefinite, the kernel function $K$, it is \textbf{necessarily} a kernel.
\item $K(x,z) = -a K_1(x,z)$\\
It's easy to see, for kernel matrices, $K = a K_1$.\\
i) Because $\forall z \in \mathbb{R}^n, z^T K_1 z \geq 0$, we can get $\forall z \in \mathbb{R}^n, z^T K z = z^T (a K_1) z = a (z^T K_1 z)\leq 0$.\\ So, the kernel matrix $K$ is not positive semidefinite. Of course, the kernel function $K$ is \textbf{not necessarily} a kernel.\\
Counter Example:\\
$K_1=\begin{bmatrix}1 & 0\\ 0 & 1\end{bmatrix}, a = 1, K = \begin{bmatrix}-1 & 0\\ 0 & -1 \end{bmatrix}$. K is not positive semidefinite.
\item $K(x,z) = K_1(x,z)K_2(x,z)$ \\
It's easy to see, for kernel matrices, $K = K_1 \circ K_2$.\ $\circ$ is Hadamard multiplication. Let's assume that $K_1(x, y) = \phi_1(x)\phi_1(y), K_2(x, y) = \phi_2(x)\phi_2(y)$\\
i) Because $K_1 = K_1^T, K_2 = K_2^T$, we can know $K^T = (K_1 \circ K_2)^T = K_1^T \circ K_2^T = K_1 \circ K_2 = K$. So, the kernel matrix $K$ is symmetric.\\
ii) Because $\forall z \in \mathbb{R}^n, z^T K_1 z \geq 0\ and\ z^T K_2 z \geq 0$, we can get 
\begin{align*}
    \forall z \in \mathbb{R}^n, z^T K z &= z^T (K_1 \circ K_2) z\\
    &= \sum\limits_i\sum\limits_j z_i \phi_1(x_i)\phi_1(x_j) \phi_2(x_i) \phi_2(x_j) z_j\\
    &=\sum\limits_i\sum\limits_j z_i (\sum\limits_k\phi_1(x_i)_k\phi_1(x_j)_k) (\sum\limits_p\phi_2(x_i)_p \phi_2(x_j)_p) z_j\\
    &=\sum\limits_k\sum\limits_p \sum\limits_i \sum\limits_j z_i\phi_1(x_i)_k \phi_1(x_j)_k \phi_2(x_i)_p \phi_2(x_j)_p z_j\\
    &=\sum\limits_k\sum\limits_p \sum\limits_i z_i \phi_1(x_i)_k \phi_2(x_i)_p \sum\limits_j z_j \phi_1(x_j)_k \phi_2(x_j)_p\\
    &=\sum\limits_k\sum\limits_p (\sum\limits_i z_i \phi_1(x_i)_k \phi_2(x_i)_p)^2 \geq 0
\end{align*}    
So, the kernel matrix $K$ is positive semidefinite.\\
Based on i) and ii), we can know the kernel matrix $K$ is symmetric and positive semidefinite, the kernel function $K$, it is \textbf{necessarily} a kernel.
\item $K(x,z) = f(x)f(z)$\\
$K=\begin{bmatrix}f(x_1)f(x_1) & f(x_1)f(x_2) & ... & f(x_1)f(x_m)\\f(x_2)f(x_1) & f(x_2)f(x_2) & ... & f(x_2)f(x_m)\\ \vdots & \vdots & ... & \vdots\\f(x_m)f(x_1) & f(x_m)f(x_2) & ... & f(x_m)f(x_m)\\
\end{bmatrix} = \begin{bmatrix}f(x_1) \\ f(x_2)\\ \vdots \\ f(x_m)\end{bmatrix}\begin{bmatrix}f(x_1) & f(x_2) & ... & f(x_m)\end{bmatrix}$\\
Let's assume $b = \begin{bmatrix}f(x_1) \\ f(x_2)\\ \vdots \\ f(x_m)\end{bmatrix}$. So, $K = b b^T$.\\
i)$K^T = (b b^T)^T = (b^T)^T b^T = b b^T = K$ So, $K$ is symmetric.\\
ii)$\forall z \in \mathbb{R}^n, z^T K z = z^T b b^T z = (z^T b)^2 \geq 0$ So, $K$ is positive semidefinite.\\
Based on i) and ii), $K$ is a valid kernel.
\item $K(x,z) = K_3(\phi(x),\phi(z))$\\
The kernel matrix $K$ is equal to $K_3$. So, $K$ is symmetric and positive semidefinite. $K$ is a valid kernel.
\item $K(x,z) = p(K_1(x,z))$\\
Let's assume $p(x) = \sum\limits_k a_k x^k, a_k >0$ and $A^{(k)}$ denotes the hadamard product of $k$ matrices $A$. So, $K = p(K_1) = \sum\limits_k a_k K_1^{(k)}$\\
Based on the problem e), $K_1^{(k)}$ is a valid kernel.\\
Based on the problem c) and the fact that $a_k \geq 0$, $a_k K_1^{(k)}$ is a valid kernel.\\
Based on the problem a), $\sum\limits_k a_k K_1^{(k)}$ is a valid kernel. So, $K$ is a valid kernel.
\end{enumerate}
\end{answer}
} \fi
