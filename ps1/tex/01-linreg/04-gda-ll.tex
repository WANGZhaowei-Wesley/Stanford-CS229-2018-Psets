\clearpage
\item \subquestionpoints{7} For this part of the problem only, you may
  assume $n$ (the dimension of $x$) is 1, so that $\Sigma = [\sigma^2]$ is
  just a real number, and likewise the determinant of $\Sigma$ is given by
  $|\Sigma| = \sigma^2$.  Given the dataset, we claim that the maximum
  likelihood estimates of the parameters are given by
  \begin{eqnarray*}
    \phi &=& \frac{1}{m} \sum_{i=1}^m 1\{y^{(i)} = 1\} \\
\mu_{0} &=& \frac{\sum_{i=1}^m 1\{y^{(i)} = {0}\} x^{(i)}}{\sum_{i=1}^m
1\{y^{(i)} = {0}\}} \\
\mu_1 &=& \frac{\sum_{i=1}^m 1\{y^{(i)} = 1\} x^{(i)}}{\sum_{i=1}^m 1\{y^{(i)}
= 1\}} \\
\Sigma &=& \frac{1}{m} \sum_{i=1}^m (x^{(i)} - \mu_{y^{(i)}}) (x^{(i)} -
\mu_{y^{(i)}})^T
  \end{eqnarray*}
  The log-likelihood of the data is
  \begin{eqnarray*}
\ell(\phi, \mu_{0}, \mu_1, \Sigma) &=& \log \prod_{i=1}^m p(x^{(i)} , y^{(i)};
\phi, \mu_{0}, \mu_1, \Sigma) \\
&=& \log \prod_{i=1}^m p(x^{(i)} | y^{(i)}; \mu_{0}, \mu_1, \Sigma) p(y^{(i)};
\phi).
  \end{eqnarray*}
By maximizing $\ell$ with respect to the four parameters,
prove that the maximum likelihood estimates of $\phi$, $\mu_{0}, \mu_1$, and
$\Sigma$ are indeed as given in the formulas above.  (You may assume that there
is at least one positive and one negative example, so that the denominators in
the definitions of $\mu_{0}$ and $\mu_1$ above are non-zero.)

\ifnum\solutions=1 {
  \begin{answer}
Calculate the derivative of $l(\phi, \mu_0, \mu_1, \Sigma)$ with respect to $\phi$:
\begin{align*}
    \frac{\partial l}{\partial \phi} &= \sum\limits_{i=1}^m (log P(x^{(i)}|y^{(i)};\mu_0,\mu_1,\Sigma) + log P(y^{(i)};\phi))\\
    &= \frac{1}{P(y^{(i)}; \phi)} \frac{\partial \phi^{y^{(i)}} (1 - \phi)^{(1 - y^{(i)})} }{\partial \phi}\\
    &= \sum\limits_{i=1}^m(\frac{y^{(i)}}{\phi} + \frac{1 - y^{(i)}}{1 - \phi})
\end{align*}
set the derivative to zeros:
\begin{align*}
    &\sum\limits_{i=1}^m(\frac{y^{(i)}}{\phi} + \frac{1 - y^{(i)}}{1 - \phi}) = 0\\
    &\phi = \frac{\sum\limits_i y^{(i)}}{m}
\end{align*}
Calculate the derivatives of $l(\phi, \mu_0, \mu_1, \Sigma)$ with respect to $\mu_0,\mu_1, \Sigma$:
\begin{align*}
    \frac{\partial l}{\partial \mu_1} &= \sum\limits_{i=1}^m y^{(i)} \frac{x^{(i)} - \mu_1}{\sigma^2}\\
    \frac{\partial l}{\partial \mu_0} &= \sum\limits_{i=1}^m (1 - y^{(i)}) \frac{x^{(i)} - \mu_0}{\sigma^2}\\
    \frac{\partial l}{\partial \sigma} &= \sum\limits_{i=1}^m (-\frac{1}{\sigma} + \frac{(x^{(i)} - u_{y^{(i)}})^2}{\sigma^3})
\end{align*}
Set all derivatives to zero:
\begin{align*}
    \mu_1 &= \frac{\sum\limits_{i=1}^m y^{(i)}}{m}\\
    \mu_0 &= \frac{\sum\limits_{i=1}^m (1 - y^{(i)})}{m}\\
    \Sigma &= \sigma^2 = \frac{\sum\limits_{i=1}^m (x^{(i)} - \mu_{y^{(i)}})^2}{m}
\end{align*}
\end{answer}

} \fi
